\def\mbox#1{\leavevmode\hbox{#1}}

\input epsf

\font\footnotefont=cmr7
\font\mainfont=cmr12
\font\mi=cmti12
\font\weekfont=cmbx16
\font\headingfont=cmbx14
\font\subheadingfont=cmbx12
\font\titlefont=cmbx18

\def\RCS$#1: #2 ${\expandafter\def\csname RCS#1\endcsname{#2}}

\RCS$Id: workbook.tex 2 2007-07-19 13:00:48Z pdezwart $

\def\week#1{\vskip 12 pt \leftline{\weekfont #1:}}
\def\heading#1{\vskip 12 pt \leftline{\headingfont #1:}}
\def\subheading#1{\vskip 6 pt \leftline{\subheadingfont #1:}}
\def\task#1{\vskip 6 pt \leftline{\mainfont #1}}

\def\quotation#1#2{{\mi ``#1''}~\cite{#2}}

\def\title#1#2#3{\centerline{\titlefont #1} \vskip 12pt \centerline{\mainfont Author: #2} \vskip 12 pt \centerline{\mainfont Student ID\#: #3} \vskip 12 pt \centerline{\footnotefont \RCSId} \vskip 24pt}

\rightskip=1.5cm
\leftskip=3cm
\baselineskip 18pt

\title{IACT201: Workbook, weeks 1 through 5}{Peter Nathaniel Theodore de Zwart}{9840642}

\mainfont

\week{Week 2}

\heading{Australian IPP and NPP homework}

\subheading{Australian Privacy IPPs Homework}

\task{Choose one of the IPPs and find out more about it.}

In regards to IPP 9: Personal information to be used only for relevant purposes. ``A record-keeper who has possession or control of a record that contains personal information shall not use the information except for a purpose to which the information is relevant.''

\task{Include a practical example of where it applies}

This principal helps to reduce the effect of gathering private data for no specific use. For example, if a Doctor's surgery kept information on their patients to maintain their quality of service they wouldn't be allowed under this principal to retain data not related to the provisioning of that service.

\subheading{Australian Privacy NPPs Homework}

\task{Investigate the relationship between the IPPs and the NPPs.}

The relationship between the IPPs and the NPPs is that the IPPs is for the public sector and the NPPs is for the private sector. For example, some of the NPPs are relevant for the private sector but not for the public sector.

\task{Show a mapping between the two.}

\item - IPPs 1-3,9 map to NPPs 1, 10.
\item - IPP 4 map to NPP 4.
\item - IPP 5 map to NPP 5.
\item - IPPs 6-8 map to NPP 6.
\item - IPPs 10-11 map to NPP 2.

\task{Identify and describe one major difference in the IPPs and the NPPs.}

One major difference is that the NPPs have provisions so that individuals can have the choice to have Anonymity, so that they can not be readily identified from the data collected from them.

\heading{Include article relating to privacy no more than 3 months old}

\epsffile{article.ps}

\heading{Tapscott article annotation}

\task{Bibliographical Detail.}

Tapscott, D. (1996), The Digital Economy, McGraw-Hill, New York, pp271-283.

\task{Annotation.}

Tapscott writes about potential issues with personal privacy in regards to the myriad of current problems as outlined in the article.

What is discussed are a few ways in which private data can be collected and abused as well as potential solutions to the problems raised.

The article is heavily centralised on the U.S.A. however the same problems are also looked at from the view of the O.E.C.D.

\week{Week 3}

\heading{Privacy in NSW Homework}

\task{Compare the NSW IPPs to the Federal IPPs.}

The NSW IPPs seem to be very close to the Federal IPPs, however a bit simpler. Once divergence that I noticed was the following, ``The agency can also disclose your information if it is for a related purpose and they don't think that you would object.''

\task{Identify and describe any differences.}

There also seems to be a difference in the targeted usage of the information between the NSW and Federal IPPs, the Federal IPPs tends to refer to the user of private information in publications where the NSW IPPs is concerned about the general use and storage of that information.

\task{Map the NSW IPPs against the Federal IPPs and NPPs.}

It shouldn't map against the Federal NPPs, as the NSW IPPs are for the public sector, not private.

\item - NSW IPP 1 map to Federal IPP 1.
\item - NSW IPPs 2-3 map to Federal IPP 2.
\item - NSW IPP 4 map to Federal IPP 3.
\item - NSW IPP 5 map to Federal IPP 4.
\item - NSW IPP 6 map to Federal IPP 5.
\item - NSW IPP 7 map to Federal IPP 6.
\item - NSW IPP 8 map to Federal IPP 7.
\item - NSW IPP 9 map to Federal IPP 8.
\item - NSW IPP 10 map to Federal IPPs 9-10.
\item - NSW IPPs 11-12 map to Federal IPP 11.

\heading{Triple J Surveillance comment}

The audio article on the use of CCTV surveillance, particularly focused around the Bankstown area of Sydney, discusses the mixed consequences of their use in the attempt to increase public safety/security.

In so far that the largest reason the CCTV units have been placed in popular public places is so that the local government can show that they are doing something to appease the popular view.

CCTVs tends to be a patch upon any problems that they are used to ameloriate, where the problem is ``displaced''. They are a ``situational prevention'', where the cameras do not stop the behaviour at the root of the problem.

Apparently there is no regulation on their use but the data that is gathered does come under the Privacy Act.

The listener is encouraged to think about and debate the use of CCTV, especially as the level technology increases to the point where ``Big Brother'' is possible. Where separate CCTV systems could be connected to track the movement of an individual. It is important to not become paranoid about the issue so that emotions stay out of the issue to avoid polarisation between viewpoints of the use of CCTV.

\heading{Essay Writing I and Exercises}

\subheading{Task: Rephrasing an essay question}

\task{What are the two instructional words?}

Critically analyse and Evaluate.

\task{Rewrite or rephrase the essay question.}

Australia has a new proposed data privacy legislation where the distribution of e-mail addresses and spamming may be illegal without individual consent. Critically analyse potential problems arising from such a legislation and evaluate the polar aspects of such an implementation.

\subheading{Exercises}

\task{Analyse the question. What are the key subject concepts related to this question?}

\item - Access and control of personal information.
\item - Appropriate balance between individuals and organisations.
\item - Controlled access to information.
\item - Information needed to make decisions.

\task{Identify the key instructional words.}

\item - Argues.
\item - Reasonable resolution.

\task{Rephrase the question}

Individuals and Organisations claim to have opposing needs in regards to the access of an individuals private data. Individuals need control over their private data where as Organisations vie for that information to make decisions. The existence of an appropriate balance is arguably debatable. Give an explanation on how a reasonable resolution can be achieved to meet the needs of both side of the problem.

\task{Take a tentative position on the question and plan out arguments for that position.}

That a reasonable resolution does exist. Have both parties define, up front, their desired access and control schemas so that a common ground can be sought. Perhaps on impartial governing body can be created to oversee such a method and also arbitrate disputes. Have a set of regulations that must be upheld by both parties so that trust can be curried.

\task{Take the alternative position and identify three key issues for this side.}

That no reasonable balance can be struck. The needs of the individual are mutually exclusive with the needs of the organisations. The cost of devising and maintaining a governing solution is prohibitive. Expecting and assuming that either side will act rationally is potentially flawed initial case for inductive reasoning.

\week{Week 4}

\heading{Essay writing II and III}

\subheading{Essay writing II exercises}

\task{Identify your thesis statement.}

The problem is that individuals desire to control access to their private data but this restricts the attempts of organisations to utilise this valuable resource in decision making processes. As this information flow has become a vital part of our society, such that a resolution must be sought to reasonably meet the needs of all parties concerned.

\task{Identify the outline of the argument you will use.}

This can be achieved through measures such as the establishment of a governing body to oversee such privacy matters, the drafting of a code of conduct that all parties involved agree and hold to, and the ability of organisations to self regulate themselves within the previous frameworks.

\task{Write a draft introduction.}

There is a disparity in the information world on the use of individuals private data by organisations such as government and business groups. The problem is that individuals desire to control access to their private data but this restricts the attempts of organisations to utilise this valuable resource in decision making processes. As this information flow has become a vital part of our society, such that a resolution must be sought to reasonably meet the needs of all parties concerned. This can be achieved through measures such as the establishment of a governing body to oversee such privacy matters, the drafting of a code of conduct that all parties involved agree and hold to, and the ability of organisations to self regulate themselves within the previous frameworks.  With these methods in place, there is a visible structure that empowers both the individual and the organisation to work together and share the desired information in a manner befitting the need for privacy of the individual and the need for access to private data for the organisation.

\task{Roughly plan the essay and write down four topic sentences.}

\item - Establishment of a governing body to oversee such privacy matters.
\item - The drafting of a code of conduct that all parties involved agree and hold to.
\item - The ability of organisations to self regulate themselves within the previous frameworks.
\item - There are the stand points of the bipolar societies of utopian liberetarianism and a draconian police state where in the former, the rights of the individual prevail as opposed to that of a police state where the rights of governing organisations have sway.

\task{Draft a conclusion making sure you reiterate the main points.}

Where there is a conflict of interest between the individual and their desire to protect what they deem private and that of larger organisations utilisation of such data to create more affective decisions, setting up a safe environment for the transfer and use of private data through public education, government regulation, industry codes of practice and enforceable protection of such would allow for a reasonable resolution to be attained. Bear in mind that there also exists the possibility for the exact opposite to occur if the fundamental trust, that all this hierarchy is built upon, is violated. 

\subheading{Essay writing III exercises}

\task{Check the language of your introduction written previously. Does it look more like Example 1 or Example 2.}

Example 2.

\task{If you can improve the language, rewrite the paragraph.}

This is entirely academic now, as the essay has already been submitted. So much for post-hoc improvements.

\heading{Tavani q 1-3}

\task{1. What is personal privacy, and why is privacy difficult to define?}

Initially it was the freedom from physical intrusion. Then it became associated with not having ones personal affairs interfered with. Later on, it became the access and control over ones own personal information.

It is difficult to define as there is now no physical article that is privacy, it is insubstantial.

\task{2. Why is privacy valued? Is privacy an intrinsic or instrumental value? Explain.}

Privacy is valued as it forms one of the core components of human interaction, being able to communicate and interact with others, without it, society would have no solid foundations.

It is an instrumental value, see above. It is a necessary precondition to other values. Without such a value, how would people interact with others. There would
be no trust. A relationship without trust can hardly be thought to be beneficial for all parties concerned. Society exists for the benefit of all, right?

\task{3. Old threats vs new threats.}

Data can now be transported electronically anywhere in the world, it no longer needs to be physically moved. Less effort and time is involved, thus, information can be gathered and disseminated cheaply.

\heading{Lecture 4 homework}

\week{Week 5}

\heading{Annotate article about recent ethical debate}

null

\heading{Tavani q 4-9}

\task{4.}

\item - Cookies, ability to track a users movements across the WWW.
\item - Spyware, watches what you are doing on your computer.
\item - Interception, watches the information flow between your computer and another host.

\task{5.}

Using data gathered about an individual to survey what they are doing. Rather than explicitly looking a the individual, a profile can be obtained from disjoint locations using the above methods in question 4.

\task{6.}

An Internet cookie is a file stored by a users web browser that maintains state information.

They are used by websites to identify a user, either for state maintenance as HTTP is a stateless protocol, so that their preferences can be obtained or for nefarious purposes, such as to show targeted marketing information, advertisements.

\task{7.}

Computerised merging, is the ability to merge disjoint databases to correlate matching information to create a new database with more information about it's entries.

It is controversial as it allows for more clandestine approaches for data gathering which can be correlated to produce a database with greater accuracy.

\task{8.}

Computerised matching, similar to merging, is the searching of two disjoint databases for similar information, allowing for targeted information.

This could be detrimental from the point of view of the individual where a government organisation could be well poised to persecute them based on a search criterion that requires little effort.

\task{9.}

Datamining is the use of ``intelligent'' search mechanism to extract data about an entity from a group of databases where information could turn from confidential and secure to non-confidential as the correlative techniques are able to form specific data about entities.

The same reasons as all the previous methods of data gathering also apply to this as this is just a ``smarter'' scheme of data correlation. Usually to do with set arithmetic, or fuzzy sets.

\heading{Referencing exercises}

null

\heading{Lecture 5 homework}

null

\week{Week 6}

\heading{Tavani q 10 - 16}

\task{10.}

The extent of a persons privacy in a public space, as in, if personal information is disclosed in a public forum by the individual who has that private data, it would no longer be private due to the public disclosure.

However, there is a class of information that is public knowledge but gathered together it can still be used to create profiles on an individual.

The problem is that this information is public and needs to be protected from it being abused as a form of vilification.

\task{11.}

Unauthorised information that is published on the Internet can be accessed by anyone who uses a search criterion that matches the desired data.

Problem is no the search engines, it is how the data got on a public forum in the first place. But who has the responsibility to protect such data?

\task{12.}

In my view, this question is asinine and rhetorical, ``Why were public records made public in the first place?'', they are public records, keyword being public.

What the problem is however, is that the original intent for making the information ``public'', eg: reason for collection, is not why the data was published, now the data can be used for other purposes due to it's disclosure.

\task{13.}

P.E.T.'s are mechanisms created in attempts to increase the ability of an entity to create privacy through technology. The problem is that social inequity, either through nature or nurture, poses a problem as they may be only available to those who can pay or have enough technical savvy to use them.

There is also the point that if a entity is unaware of a technology, how are they to take advantage of it.

\task{14.}

Most self regulation done by industries, national or international, are codes of practice that were enacted to either stem government legislation, allowing for greater flexibility to counteract privacy concerns. This measure should increase the public trust of organisations following the code.

The problem is that they are enacted by the very organisations who desire access to private data, so there seems to be a conflict of interest here, limiting their effectiveness.

\task{15.}

\item - SECTION I - PRINCIPLES RELATING TO DATA QUALITY
\item - SECTION II - CRITERIA FOR MAKING DATA PROCESSING LEGITIMATE
\item - SECTION III - SPECIAL CATEGORIES OF PROCESSING
\item - SECTION IV - INFORMATION TO BE GIVEN TO THE DATA SUBJECT
\item - SECTION V - THE DATA SUBJECT'S RIGHT OF ACCESS TO DATA
\item - SECTION VI - EXEMPTIONS AND RESTRICTIONS
\item - SECTION VII - THE DATA SUBJECT'S RIGHT TO OBJECT
\item - SECTION VIII - CONFIDENTIALITY AND SECURITY OF PROCESSING
\item - SECTION IX - NOTIFICATION

Overall, the strengths and weaknesses of the directive in general is in the fact that data is not secret but private, data can be collected but not used without individual consent.

\task{16.}

See essay, redundant reiterating that here.

Outline: Educate the public, make them aware of what the issues are and what is available to them to mitigate the problems. Same as dealing with any other problem, find the cause, eg: greed, solve cause, purge humanity. Reasonably solution may be intractable due to the complexity of human society. Working solution may be possible with some ``acceptable'' errors.

\heading{Annotate article related to essay 2}

null

\heading{Editing exercise}

null

\bye
