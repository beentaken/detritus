\def\mbox#1{\leavevmode\hbox{#1}}

\input epsf
\input btxmac

\font\footnotefont=cmr7
\font\mainfont=cmr12
\font\mb=cmbx12
\font\mi=cmti12
\font\weekfont=cmbx16
\font\headingfont=cmbx14
\font\subheadingfont=cmbx12
\font\titlefont=cmbx18

\def\RCS$#1: #2 ${\expandafter\def\csname RCS#1\endcsname{#2}}


\def\week#1{\vskip 12 pt \leftline{\weekfont #1:}}
\def\heading#1{\vskip 12 pt \leftline{\headingfont #1:}}
\def\subheading#1{\vskip 6 pt \leftline{\subheadingfont #1:}}
\def\task#1{\vskip 6 pt {\mi #1}}

\def\quotation#1#2{{\mi ``#1''}~\cite{#2}}

\def\title#1#2#3{\centerline{\titlefont #1} \vskip 12pt \centerline{\mainfont Author: #2} \vskip 12 pt \centerline{\mainfont Student ID\#: #3} \vskip 12 pt \centerline{\footnotefont \RCSId} \vskip 24pt}

\leftskip=0.75cm
\baselineskip 24pt

\title{IACT201: Workbook, weeks 7 through 12}{Peter Nathaniel Theodore de Zwart}{9840642}

\mainfont

\week{Week 7}

\heading{Tavani, questions 1 through 12}

\subheading{Question 1}

\task{What is intellectual property?}

Intellectual property consists of objects that are not tangible, it is not the ownership of an ephemeral idea, however they are the expression of an idea in to a tangible form. For example, composition of an idea in to a written media, such as a book, or the development of an invention, such as the combustion engine.

\subheading{Question 2}

\task{How is intellectual property different from tangible property?}

Intellectual property is different from tangible property in the sense that an entity owns the expression of a intangible idea in to a tangible form, for example, the process of manufacturing a certain pharmaceutical, rather than owning tangible property, for example, the plant machinery to manufacture the previously mentioned pharmaceuticals would be a owned tangible property.

\subheading{Question 3}

\task{What is meant by an ``intellectual object''?}

An ``intellectual object'' is a manifestation or expression of an idea, they are also non-exclusionary, meaning that copies can be concurrently owned by different entities.

\subheading{Question 4}

\task{Do intellectual objects deserve legal or normative protection?}

Intellectual objects deserve legal or normative protection. What is being protected depends on societies ideals, for example, a utilitarian society would be protecting innovation through granted rights to the owners of intellectual objects where as societies such as the European Union are protecting the owners right to the expression of their intellectual objects. Thus the reason for protection would vary dependant on the underlying social structure.

\subheading{Question 5}

\task{Describe the difficulties that arose in determining wheather computer software should be eligible for the kinds of legal protection (i.e., copyrights and patients) that are typically granted to authors of creative works.}

Initially, there were conceptual difficulties in determining what type of intellectual object computer software was, as there is a difference in the program source code and the compiled object code, as the former consists of lines of human readable instructions similar to that of a composed music work or mathematical formulae as opposed to the latter which is machine readable so the computers CPU could execute it. Thus, software programs seemed like inventions that could be patented however since the resembled these mathematical ideas, they would not usually have been granted such protection but in the end they were.

\subheading{Question 6}

\task{Describe some differences between the legal schems for protecting intellectual property: copyrights, patents, trademarks, and trade secrets.}

Copyright law pertains to legal protection granted to a ``person'' or author of the expression of an idea, such as a book, music, dance or computer software. Usually, such a work must be original, nonfunctional and fixed in a tangible medium.

Patent protections are granted to individuals who have created a process or invention, such as Edison's lightbulb. Patents offer a twenty-year exclusive monopoly over such an expression or implimentation of a protected work. Patents can be applied to such processes or inventions that are devices such as machines, articles of manufacture or ``compositions of matter''. The Patent Act requires that inventions satisfy the following: usefulness, novelty, and nonobviousness.

A trademark is a word, name, phrase or symbol that identifies a product or service. For example, a companies logo. Such trademarks can be registered once such symbolism is first used which can then be used to protect the identity of such products or services.

Trade secrets are information that is sufficiently valuable and secret to afford an economic advantage over others where they are used in the operation of a business or other enterprise. Owners of a trade secret have exclusive rights to make use of it, as long as such information remains a secret.

\subheading{Question 7}

\task{What is the Digital Millennium Copyright Act, and what are its implications for the future of copyright protection in cyberspace?}

The Digital Millennium Copyright Act extended the amount of time a copyrighted work is protected and contains a controversial anti-circumvention clause. There are many critics of the DMCA due to the anti-circumvention clause, as this has forbids the development of any software or hardware technology that circumvents copyrighted digital media. This has implications on the ``fair-use'' principal of copyright.

\subheading{Question 8}

\task{Whare are the doctrines of fair use and first sale?}

Fair use is the limited use of another person's copyrighted wors for purposes such as criticism, comment, news, reporting, teaching, scholarship, and research by an author or publisher. Thus allowing someone the ability to purchase a product for the use of ``reverse engineering''.

First sale applies when an original work is sold for the first time, the original owner loses rights over the work of art, such that the current owner may do as they see fit with the article, such as give away, resell, dispose of, etc $\ldots$

\subheading{Question 9}

\task{How is the priciple of fair use, as illustrated in the Eldred and Sklyarov cases, threatned by recent changes to copyright law?}

The power of the recent copyright law changes gives previously expired copyrights extended duration, even if such works have already been passed in to the public domain through expiry. This is problematic as there are current uses of such previously public domain works which have now moved back in to copyright.

Another problem is where fair use and first sale is being undermined with software products, such as the Sklyarov case, where decryption software was being carried in to the United States of America and the government felt that it needed to test out the DMCA. In the case of a printed book, once first sale has been completed the purchaser can do whatever they desire with the article within reason, however, the DMCA challenges this due to the need for encryption/decryption of the copyrighted works.

\subheading{Question 10}

\task{What property rights issues surround the Napster case?}

The fair use property rights issues were being challenged in the Napster case as the RIAA has accused Napster of illegally distributing copyrighted information whereas Napster defends that it was doing this legally through the fair use doctrine.

\subheading{Quesion 11}

\task{How were the controversies involving the Napster dispute anticipated by the 1994 case involving MIT student Rober LaMacchia?}

As LaMacchia was not able to be indicted for his involvement in the Cynosure BBS pirated software sharing, the 1986 Computer Fraud and Abuse Act was ammended to broaden the scope of criminal behaviour that could be prosecuted under it, and the No Electronic Theft (NET) Act was passed in 1997.

\subheading{Question 12}

\task{What are the arguments for and against protecting software with patents?}

The arguments against protecting software with patents is that software consists of mathematical formulae or abstract processes which are excluded from the patenting process.

The arguments for protecting software with patents are commercially based and not related to the philosophical reasons for the granting of copyright on intellectual objects.

\week{Week 8}

\heading{Tavani, questions 13 through 16}

\subheading{Question 13}

\task{Describe the three philosophical property theories that we considered.}

The Labor Theory of Property is where one is entitled to the fruits of their labour. This is in relation to physical labour where people ``own their own bodies''. Thus, slaves, not owning their own body, had no right to the fruits of their labour.

The Utilitarian Theory of Property is based upon the use of granted rights to further innovation for the good of society.

The Personality Theory of Property is different to the aformentioned theories in the sense that they discussed external reasons for the granting of rights, usually on an economic basis, where as the Personality Theory of Property is based upon internal factors for an entity to create a work and thus grant rights on the investment of the personality of the creator in their work. As there is believed to be a connection between the intellectual object and the ephemeral being of the creator it is desired to have protective measures to ensure that this is not abused.

\subheading{Question 14}

\task{How can each of the three philosophical theories of property be extended to intellectual property claims? Which theory seems most plausible in the Internet era?}

The Labor Theory applies in the sense that rights should be granted on the expenditure of efford by an entity to create such intellectual property.

The Utilitarian Theory applies so that innovation within society is protected.

The Personality Theory applies where Intellectual Property is seen as a work of art rather than purely an object of economy.

The theory of best fit would depend on ones viewpoint, if one would prefer economic factors, from the individual then the Labor Theory, from the greater society the Utilitarian theory. However, if one looks on Intellectual Property from an academic or artistic point of view, then the Personality Theory would be the theory of choice.

For a succinct comparison of the three theories, see table 8-2, p217 in Tavani.

\subheading{Question 15}

\task{What do we mean by the expression ``Information wants to be shared''?}

The sharing of information, particularly in academic sphers has led to many technological advances, for example, the creation of the Internet was enabled through the sharing of information. The expression is in regards to how, as humans, we desire to share information to enable us to make cultural advances rather than personal economic advances, however there needs to be a balance to ensure that we don't slip in to a state of lassitude where there is no incentive for the cultivation of intellectual property.

\subheading{Question 16}

\task{How is the expression ``Information wants to be shared'' different from the position ``Information wants to be free''?}

In the case of free information, all information should be able to be shared at no economic cost to the receipient as opposed to the sense that sharing information with economic cost. In both cases information is shared but there is a dichotomy in the sense that information is treated as having economic value and can thus be traded.

\heading{Library exercises (found on Week 8 Checklist)}

\subheading{Select a scholarly reference from Safari Tech Books Online related to the Report question}

Smart Cards: The Developer's Toolkit, By Timothy M. Jurgensen, Scott B. Guthery

\subheading{Include a printout of the first page of your search results showing your search terms}

\epsffile{safari.ps}

\subheading{Include the full correct citation for the e-book reference you have chosen}

Please see the reference section at the end.\cite{sctdt}

\week{Week 9}

\heading{True Computer crime (N3C) annotated bibliography}

\subheading{Read and write an annotated bibliography entry for the article ` ``True'' Computer Crime' by the NW3C}

Please see the reference section at the end.\cite{tcc}

The NW3C article discusses ``True'' computer crime by defining what computer crime is and the various types. It then goes in to details about the three main types of computer crime. Unauthorised access, where a person gain access to a system either by elevating their existing privledge or by gaining inital access. Theft of data and services. Sabotage of a computer system, through the use of virii, worms, denial of service attacks, etc $\ldots$

Methods of computer crime prevention are also discussed, from methods such as antivirus software, network firewalls. The costs and statistics relevant to computer crime and some examples are also delt with.

Legislation and current efforts to balk computer crime is the final component of the article.

\heading{Quirk 15.2, 5}

\subheading{Question 1}

Firstly, the is the quandry of obtaining the physical evidence off the CDs, as the removal of the blood and fingerprints must be done first and in a way not to affect the examination of either of the materials. Potentially, the blood could be sampled first, then the fingerprints dusted and removed, then by using some sort of cleaning process, usually involving isoproply alcohol and a non abrasive, absorbant material, dissolve and wipe off any foreign objects from the CDs.

Once this has been achieved, the blood, fingerprints and CDs can be sent off for analysis. Correlations could be made between the DNA in the blood and fingerprints by crossreferencing databases of known samples to link either the owners of the samples or social networks of contacts.

\subheading{Question 2}

There are a few obsticals in the gather of digital evidence.

\item{-} Digital evidence can be hidden by encryption, steganography and passwords, increasing the resources needed to persue a conviction.
\item{-} Evidence can be hard to find in the masses of data, even in a household PC. There we have a ``needle in a haystack'' predicament.
\item{-} The ease of destruction of evidence, either intentionally by the perpetrator or inadvertently by the law enforcement officers in an attempt to gather evidence.
\item{-} Digital evidence can be spread across networks that span the globe, making tracking the evidence down a resource intensive operation.
\item{-} Evidence must be preserved in order to allow for easy reproduction in a courtroom and to show an unbroken ``chain of evidence''.
\item{-} Digital evidence can be easily lost or altered in a similar fashion to how it can be destroyed in the gathering process.
\item{-} Digital evidence usually needs a lot of input by expert witnesses so that the members of the court can understand what they are viewing. If the jury can not understand the evidence than it is of no use.

\week{Week 10}

\heading{Tavani q 1 - 14}

\subheading{Question 1}

\task{Is there a typical cybercriminal?}

Yes, but a distinction has been made between the amature cybercriminal, such as a script kiddie, disenchanted employee and the professional cybercriminal.

\subheading{Question 2}

\task{Can a meaningful distinction be drawn between hacking and ``cracking''?}

Yes, hacking is typically the benign effort of willingful individuals to find faults in systems in the effort to strengthen them not for personal gain whereas cracking is seen as the delinquent attempts to gain entry in to systems either for defacement, the online form of grafitti, to that of industrial espionage.

\subheading{Question 3}

\task{Can certian forms of hacking be protected by the Constitution?}

There are advocates that push the point that hacking is an expression of individual freedoms under the first ammendment.

\subheading{Question 4}

\task{Why has the media attention given to hackers made it difficult for law enforcement authorities to track down professional criminals why use cybertechnology?}

The media is well known for aggrandisement, apply this to the perception of hackers and then a mythos is developed. Perhaps the large volume on non professional criminal activity that uses cybertechnology gets the attention and the proffessionals remain under the rader.

\subheading{Question 5}

\task{Identify some laws that have been drafted to combat crime in cyberspace.}

The DMCA, the Patriot Act.

\subheading{Question 6}

\task{What problems of jurisdiction pose for understanding and prosecuting crimes commited in cyberspace?}

The perpertrator could have been commiting the criminal act in a different location than the system that they violated. This is especially perternant when the perpertrator is in another country who might different or no legislation for cybercrime.

\subheading{Question 7}

\task{What exactly is cybercrime? Can a coheraent definition of cybercrime be framed?}

Cybercrime is a crime where the criminal act can be carried out only through the use of cybertechnology and can only take place in the cyberrealm.

Piracy, tresspass and vandalism in cyberspace are all definitions of cybercrime.


\subheading{Question 8}

\task{How can we distinguish between genuine cybercrimes and ``cyberrelated'' crimes?}

A cyberrelated crime is a crime that can be carred out in with the use of cybertechnology but is not restricted to the cyberrealm. Thus the use of cybertechnology is use to exacerbate the crime or in used in assistance.

\subheading{Question 9}

\task{How might we distinguish between ``cyberexercabated'' and ``cyberassisted'' crimes?}

Obviously, cyberexercabated crimes are those where the use of cybertechnology has exercabated the extent of the crime, whereas cyberassisted crimes are where cybertechnology has been used to assist with the perpertration of the crime.

\subheading{Question 10}

\task{Which ethical questions does the use of encryption technology raise with respect to computer crime and computer criminals?}

The questions can be divided in to four sections: security and reliability, trust in government, economic impact, and implications for civil liberties.

\subheading{Question 11}

\task{Describe some national and international efforts currently used to combat cybercrime.}

The U.S. Government enacted the National Information Infrastructure Protection Act in 1996 to ammend the Computer Fraud and Abuse Act. The Council of Europe released a first draft of the COE Convention on Cybercrime, 2000.

\subheading{Question 12}

\task{What are some problems in enforcing national and international crime laws and treaties?}

Jurisdiction is a major problem, cross border crimes are difficult to track and prosecute. There are ethical questions of the violation of cultural or constitutional rights.

\subheading{Question 13}

\task{Describe some of the controversies associated with strong encryption technologies, such as the Clipper Chip, in combating crime.}

The proliferation of strong encryption has led to the belief that cybercriminals will use such a technology to safeguard their communication channels which the law enforcement bodies would not be able to tap in to within any reasonable time.

The idea of the Clipper Chip, is to place this chip in all communications devices to circumvent the encryption, which is intractable, but let's forgive government bodies for idocy as they are well known for it.

Some other ideas was the use of software back doors to be used of government law enforcement bodies and help by software conglomerations.

All these ideas lead to the undisclosed aberration of civil liberties and constitutional rights.

\subheading{Question 14}

\task{How can biometric technologies be used to fight cybercrime and cyberrelated crimes?}

The use of biometric technologies allows for the identification of individuals in a digital media. Such identification would aid in the security of a system by stronger means of authentication but does little in the forms of protection from social engineering.

Chapter 7 seems to tote the use of biometric technologies as a ``good'' solution to cybercrime but the examples given have very little to do with the cyberrealm but more to do with authentication in the physical realm, such as airports and the ``superbowl''.

\week{Week 11}

\heading{Learning Development Critical Thinking 1 Exercises}

\subheading{Do you think these exerpts prove the existence of black holes?}

No, these statements are hypotheses and observations of a gendanken (thought) experiment, it leads to believe that the best solution would be a black hole and the application of Occam's razor would show this as well, however this is not a complete proof in the terms of a deductive or inductive proof.

\subheading{Determine the status of each sentence}

\task{1. Near galactic centres, stars are moving so rapidly that they would fly off unless the gravity of a huge mass - up to the equivalent of a billion suns - held them in.}

This is a hypothesis.

\task{2. Whatever has this mass must be extremely dense.}

This is a hypothesis.

\task{3. Theorists know of no alternative to a black hole.}

This is an observation.

\task{4. Many galactic centres and binary star systems spew radiation and matter at gargantuan rates.}

This is an observation.

\task{5. They must contain an extraordinary efficient mechanism for generating energy.}

This is a hypothesis.

\task{6. In theory the most efficient engine possible is a black hole.}

Evidence for the existance of black holes.

\subheading{Analysing assignment questions}

\task{In relation to safeguarding citizen's rights (for example as regards provacy), how important is the government's role?}

{\mb Topic area:} Citizen's rights.

{\mb Focus:} How important is the governments role.

{\mb Instructional word:} Relation.

\task{The international nature of the Internet causes problems for governments throughout the world attempting to legislate how to combat illegal activities through computer networks. What are some of these problems and how realistic are the methods being proposed to deal with them?}

{\mb Topic area:} International legislation for crime on the Internet.

{\mb Focus:} Problems and realalism of porposed methods for dealing with the topic area.

{\mb Instructional word:} Identify.

\subheading{Research one Privacy Enhancing Technology}

\task{Proxies and firewalls}

``Proxy servers and firewalls are technologies that typically are located between the individual consumer and the Internet. In a corporate environment, they may be located on the local area network (LAN) at the point where the LAN is connected to the Internet, at the ISP, or somewhere in between. Proxies and firewalls can also greatly enhance security in a network environment.

Firewalls and proxies are quite similar in terms of their functionality, though firewalls typically include additional security features not found in proxy servers.15 Generally, however, both can prevent the disclosure of an individual's IP address or other personal information by acting as an intermediary between a web site and an individual computer.

The key difference between firewalls and proxy servers is how they deliver information to an individual browser. Information requested through a firewall  whether it be a web page or streaming video is delivered directly back to the individual user. The firewall may scan for viruses, restrict certain types of content or implement additional security features, but the information is sent back to the individual computer that initially requested the data.

In a proxy environment, the proxy server acts on behalf of the individual user and hides the identity of the client computer from the web site. When an individual requests a given web page  www.oecd.org, for example  he or she is actually passing the request to the proxy, which in turns makes the request to the actual OECD web server. The OECD web server, in this example, would deliver the page and information back to the proxy, which in turn delivers the page to the individual user.

These technologies are widely deployed on corporate networks. They are readily available, often bundled with network, web site and other Internet products and services. Firewalls for individual PCs are also widely available on the retail computer software market. Because these products were originally developed for security purposes, their functionality and flexibility are often not as robust as other products developed specifically to address privacy issues. However, their widespread usage and deployment ensure that they will remain at a minimum a crucial element of any privacy-enhancing technology solution.

Proxies and firewalls are widely available from computer security firms, and are often bundled with network or web software.''\cite{oecd}

\subheading{Chirillo Chapter 1 Summary}

\task{Summary}

``Many forms of biometric systems exist for indentification and verification purposes; each has a different price range with associated crossover error rates and user-acceptance levels. This book disects these systems and formulates a cookbook-style template for your own applications. In addition, it formulates methodologies and examines object-oriented source code for srong authentication solutions. Finally, it looks at the weaknesses of each solution and how to mitigate those weaknesses to enhance security and risk acceptance in your environment - wheather it is a small home office, a medium-sized infrastructure, or a vast enterprise.''\cite{chirillo}

\week{References}
\nocite{*}
\bibliography{workbook2}   % specify the database files
\bibliographystyle{plain}   % specify plain.bst as the style file

\bye
