
\def\mbox#1{\leavevmode\hbox{#1}}
\input btxmac

\font\footnotefont=cmr7
\font\mainfont=cmr12
\font\mi=cmti12
\font\subsectionfont=cmbx12
\font\sectionfont=cmbx14
\font\headingfont=cmbx16
\font\titlefont=cmbx18

\def\RCS$#1: #2 ${\expandafter\def\csname RCS#1\endcsname{#2}}

\RCS$Id: essay1.tex 2 2007-07-19 13:00:48Z pdezwart $

\def\heading#1{\vskip 12 pt \leftline{\headingfont #1}}

\def\quotation#1#2{{\mi ``#1''}~\cite{#2}}

\def\title#1#2#3{\centerline{\titlefont #1} \vskip 12pt \centerline{\mainfont Author: #2} \vskip 12 pt \centerline{\mainfont Student ID\#: #3} \vskip 12 pt \centerline{\footnotefont \RCSId} \vskip 24pt}

\rightskip=1.5cm
\leftskip=3cm
\baselineskip 18pt

\title{IACT201: Essay 1}{Peter Nathaniel Theodore de Zwart}{9840642}

\mainfont

There is a disparity in the information world on the use of individuals private
data by organisations such as government and business groups. The problem is
that individuals desire to control access to their private data but this
restricts the attempts of organisations to utilise this valuable resource in
decision making processes. As this information flow has become a vital part of
our society, such that a resolution must be sought to reasonably meet the needs
of all parties concerned. This can be achieved through measures such as the
establishment of a governing body to oversee such privacy matters, the drafting
of a code of conduct that all parties involved agree and hold to, and the
ability of organisations to self regulate themselves within the previous
frameworks.  With these methods in place, there is a visible structure that
empowers both the individual and the organisation to work together and share
the desired information in a manner befitting the need for privacy of the
individual and the need for access to private data for the organisation.

Even given the above, there is still a flip side to this standpoint, the
situation where there is no reasonable resolution. For example, there are the
stand points of the bipolar societies of utopian liberetarianism and a draconian
police state where in the former, the rights of the individual prevail as
opposed to that of a police state where the rights of governing organisations
have sway. In these two circumstances, sublime as they are, there would be no
reasonable resolution. However, these socio economic climates are rare at best
with most populaces political agendas lying in between the two, with the hope
that they are closer to the centre than the outlying borders. It could be
thought that the greater balance that a society has between the needs of the
individual and that of the whole, would have a greater chance of reaching this
reasonable resolution of conflicting needs.

The greater problem that would oppose a solution is that of confidence in the
processes employed by organisations seeking access to private data by the
individuals willing to share this sensitive information.  In the past, some
large businesses that deal with personal information gathered on the Internet
have been under the spotlight in the United States, as Dreazen wrote about
the probe in to the information collection practices of DoubleClick, an online
advertising provider, \quotation{...one of the biggest online advertising
firms, ...has been accused of violating the privacy of Internet users by
profiling them with data collected from cookies and other
measures.}{twsj-guard} and where Richtel wrote about Comcast ceasing to store
data on it's users Internet surfing habits, after pressure from the US
Congress and privacy groups\cite{nyt-csiwssdoc}.

Issues such as these tend to skew the view of the public from volunteering
their private data to organisations who would use the information in a
reasonable manner. To assuage this, methods have to be put in place such
that abuse can be identified and controlled. Tapscott identified four available
approaches to the issue of privacy: Voluntary codes and standards, regulation,
consumer education and technology solutions\cite{tde}. However Showalter and
Turinas believe that \quotation{Technology and market forces have limited
effectiveness}{mip-priv-elec-age} as security given by technology does not
protect the individual from all forms of disclosure and voluntary regulation
is limited in the sense that it is voluntary. However, there is awareness in
the marketplace where \quotation{companies are taking a hard look at their
privacy policies...much more careful about letting customers know what they are
doing.}{wsj-ecom} but even this may not be enough.

What is needed is a combined approach where all four methods are employed. As
is done in Australia, where there is an established governmental office, the
Federal Privacy Commissioner, \quotation{whose job it is to administer [The
Commonwealth Privacy Act 1998] as well as to promote privacy, give policy
advice to government, monitor compliance and investigate complaints.}{itcr}.
This act and office gives individuals statutory protection when dealing with
government organisations, where non-compliance would afford the individual
with adequate compensation for any violation of the Information Privacy
Principals. This environment will allow individuals to be comfortable with
dealing with governmental organisations in their private information. However,
this doesn't address the private sector. Private Sector organisations must
adhere to when collecting data from an individual, the ten National Privacy
Principals, these principals are based on the Information Privacy Principals.

The de facto view of the private sector in regards to the privacy interface has
been to self regulate. Relying on market forces to guide businesses is
generally not a good thing due to their tropistic propensity towards profits.
Evidence of this can be see in the philandering of various article writers
for journals that represent industry areas such as direct marketing and
insurance. It can be seen that such greedy corporate values influence the
market to abuse any advantage available to gain a larger market share and thus
appease their shareholders. Robertson and Sarathy wrote, \quotation{Egoism is an
ethical philosophy in which individuals, or firms, justify their actions based
on their own self interests...The moral growth of individuals who are involved
with making strategic decision about privacy issues is also a factor influencing
where the firms end up making privacy protection decisions.}{jbe-secmdp},
showing that, if left up to the corporations to self regulate, there is reason
to believe that they will not regulate appropriately in view of the problem of
private data sharing between individuals and organisations.

In light of this, there needs to be an enforceable code of practice that all
private organisations agree to adopt when they require access to an individuals
private data to ensure that public trust is not violated as part of their
social contract.  With that, an industry ombudsman backed up by the Federal
Privacy Commissioner to give them some clout to enforce the code. Without
realistic penalties for transgressions, there would be no value in the code.
There are such codes available in Australia, organisations such as the Internet
Industry of Australia impose soft laws on Internet Service Providers with some
help from the Australian Broadcasting Authority, although there is public
discussion on how the ABA relates to a non broadcast medium.

The problem also has to be approached from the point of the individual as well.
Public education is lacking when it comes to privacy and what peoples rights
and responsibilities are. Educational campaigns aimed at the uninformed about
the dangers of privacy breach would create a greater awareness of the situation
and what measures can be put in place to create a safer environment for
individuals.

Once such an environment is set up, with the discussed structures erected for
both public and private sector organisations, there would be a greater flow of
information where individuals would be comfortable in having their private data
utilised safely by organisations. This in turn would generate aggregated
information of a higher degree of accuracy allowing civic minded organisations
to make mindful decisions. All this is based on trust however, like inductive
reasoning, if the initial assumption is false, the rest of the proof falls in
a heap. Likewise, if the trust of the individual is shattered at any of the
levels in the privacy control hierarchy, there would be a less willing attitude
of private data sharing. Such a situation could become pathological, where a
breach in trust would cascade to a restriction in the flow of data which would
lead to clandestine information gathering schemes, which would feed in to this
situation once knowledge of such methods are revealed.

On the other hand, evidence of trustworthy behaviour on the part of
organisations would show individuals that there is a solid privacy foundation
that they can rely upon, where the decisions made benefit the one and the many
due to the greater availability of data.

Where there is a conflict of interest between the individual and their desire
to protect what they deem private and that of larger organisations utilisation
of such data to create more affective decisions, setting up a safe environment
for the transfer and use of private data through public education, government
regulation, industry codes of practice and enforceable protection of such would
allow for a reasonable resolution to be attained. Bear in mind that there
also exists the possibility for the exact opposite to occur if the fundamental
trust, that all this hierarchy is built upon, is violated. 

\heading{References}

\nocite{*}   % put all database entries into the reference list
\bibliography{essay1}   % specify the database files
\bibliographystyle{plain}   % specify plain.bst as the style file

\bye
