\section{Feasibility study}

\subsection{Interests:}
A University would be interested in the application of a stored value card
system within and around the grounds because of the huge benefits it can
provide to both staff and students. The benefits of the stored-value card
system can be outlined by: 

\items High level of security: resulting from advanced levels of user or
terminal authentication, standardised use of PIN and passwords to eliminate
chances of misuse. Our system provides you with a complete audit trail for all
card holder and merchant accounts. It also gives you the ability to track
purchasing habits to better understand your clientele and target your marketing
efforts on the most profitable customers.

\subitems Reduces the risk of theft. 

\items Convenience: quicker transactions, no need for notes or coins. 

\subitems Students like the convenience of using the card without having to
worry about needing cash on hand. 

\subitems The cards can also double as student ID cards, eliminating the need
for students to carry and keep track of multiple cards. 

\subitems Storage of personal data, e.g. medical records.

\items Cost effectiveness: easy to maintain and assess. 

\subitems Students like the fact that a lost or stolen card can be cancelled.
Any unspent value on the card can be transferred over to the replacement card.
Your organisation sets the card replacement policy.

\subitems Minimise new student processing time.

\items Multi functional: can be used for a wide variety of functions.

\subitems Parents like the peace of mind of knowing that funds placed on a
stored-value card cannot be spent for things that it was not intended for. The
stored-value cards can only be used at merchants that are a part of the
program. Therefore, your organisation controls the scope of the program.
Agreements with participating merchants can also specify items that the card
cannot be used to purchase.

\subitems The program can be limited to on-campus locations or expanded to
include independent, off-campus merchants.

\subitems We can work with you to create multiple "card products" so that you
can separate your clientele into groups such as faculty, staff, students,
tutors, etc $\ldots$

\subsection{Applications:}

Applications for the stored value card throughout and around the university
grounds would include such things as:

\items Alternative to, or replacement for cash: termed electronic purses. The
cards could be linked to the individuals account to the university. The account
would be either a debit or credit to an applied limit. Purchases could be
automatically deducted form that person's account at the place of purchase at
real time and an accurate account of all purchases via a statement produced
monthly or quarterly. The cards could also be linked to other shops in and
around the campus such as local businesses.   

\subitems Student and Staff security passes to buildings and computers. For
rooms with expensive equipment such as computer rooms and media labs this would
be effective in reducing theft especially during later hours. This application
would also be used to restrict access to unauthorised students. For example not
allowing students into staff areas. Restricting access to computers in cases
where the student is not part of the faculty or school.      

\subitems Library card. Stored value cards would replace the current library
card, allowing students to access the library, place books on hold, reserve
books, loan books, access and use the photocopying machines and printers. All
with the one card.

\subitems Roll call. Cards could also be used to tract student attendance to
both lectures and tutorials. Lecturers could easily track attendance and
amount of time spent in lecturers to help fairly award participation marks and
help in determining scores in borderline mark cases.    

\subitems Paying for car parking. The card could allow access and track
parking and allow the user to only pay for the parking that is used at a fair
rate. 

\subitems Loyalty and shopper discount program cards, the card would be able to
track university usage at shops and develop loyalty programs for services that
are used by students. 

\subitems Storage of personal data, e.g. medical records, this could be used in
emergencies and could greatly help authorities in cases where the information
provided could save someone's life. 

\subitems Storage of an individual's personal and biometric information such as
facial geometry, finger/thumb prints, voice patterns, identifying
characteristics, electronic signatures, written signature, and so on;

\subitems Storage of a person's user interface preferences e.g. speech or
large print display on an Automatic Teller Machine.
