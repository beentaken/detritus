\section{Implications of stored-value card implementation}

The use of stored-value cards has a number of implications, relating to both individual privacy and technology. Before such a system is implemented, it is important to give consideration to these potential problem areas. Depending on the specific details of the card's implementation, the extent of these implications may vary. If the University implements the idea to store a significant amount of an individual's information on the cards, it becomes significantly more important to ensure that the system itself, and the results of its implementation, are carefully controlled.

One of the foremost concerns arising from the implementation of a stored-value card system is in relation to individual privacy. Smart card systems are capable of bringing about both positive and negative changes in relation to privacy, with the extent of such changes being determined by the specific details of the implementation \cite{clarke}. 

It is suggested by some that value-stored cards in fact increase an individual's privacy, by allowing for sensitive information to be stored in a digital format, allowing for the use of various data protection technologies \cite{everett}. One significant flaw in this argument is that such systems can be compromised, meaning that the privacy of an individual's information stored in such a card can not be guaranteed \cite{kanninen}. This in turn means that individuals within the University may be hesitant to adopt the new system, as the security of their data is not ensured.

To further compound this issue, many argue that a value-stored card system in fact breaches an individual's privacy, simply by storing their personal information. It is suggested that the belief that value-stored cards can in fact increase privacy by offering data protection violates the two separate notions of security and privacy \cite{clarke}. 

This issue has been a significant concern during the implementation of such systems in other tertiary education environments. For example, during 1997, a stored-value card system was introduced at the University of Toronto, Canada. Many students at this institution were particularly concerned about the collection of statistical data about their activities, made possible by the new stored-value card system \cite{bozak}.

As a stored-value card system by its very nature facilitates this form of information gathering, it is likely that similar concerns would arise in relation to the implementation of the system at this University. Such implications would need to be carefully considered prior to the implementation of such a system, and measures would need to be taken to ensure that student acceptance of the new system was positive.
 
Another issue related to privacy which arises as a result of the implementation of a stored-value card system is that of ``function creep'' \cite{bozak}. It has been suggested that in our current society, initially many proposed value-stored card systems are not financially justifiable based on their proposed uses. In order to make these cards more cost-effective, as time goes on numerous features are added, resulting in the card storing far more information than was initially intended. In terms of privacy this can have very serious implications, as often this new information is sold to data mining organisations in order to minimise the cost of the system. Concerns relating to this issue are well established, meaning that any University considering implementing such a system could find user acceptance to be an issue \cite{connoly}.

These two conflicting viewpoints ensure that any stored-value card system which is implemented within the University will bring about privacy issues which will need to be dealt with effectively in order to ensure that the system is successfully adopted. 

Another issue which arises as a result of the implementation of a stored-value card system is that of technology. As no such system is currently in place, hardware and software will need to be purchased and installed in order to support the value-stored card system. University employees who will be working with the value-stored card system will also need to be trained in order to minimise transitional problems. Such training will also maximise the efficiency of the new system, making it a valuable addition to the University.

Aside from the initial implementation issues associated with the implementation of this system, a number of ongoing factors will also need to be considered. For example, once the system has been implemented it is will require periodic maintenance. 

The introduction of a stored-value card system also has implications relating to technology failure within the University. For example, an institution relying heavily upon a value-stored card system may experience significant difficulty continuing operation as normal, unless adequate precautions have been taken. Such precautions could include keeping the existing system in place, while at the same time also implementing the stored-value card system.

One potential flaw in this solution is that while it ensures that in the event of a failure in the stored-value card system the University will still be capable of operating as normal, at the same time it increases operational costs. This is due to the fact that both systems need to be maintained and operated in conjunction for this solution to adequately resolve the problem.

Overall it is obvious that the implementation of this system will result in a number of ramifications which the University must be prepared to deal with. While the exact nature of these implications will vary depending upon the specific design of the implemented value-stored card system, two of the issues which are likely to arise are in relation to individual privacy and technology. To successfully deal with these issues, the University must carefully examine the proposed value-stored card system and ensure that each of the potential problems is carefully addressed in both the planning and implementation stages. 
