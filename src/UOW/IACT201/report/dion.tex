\section{Advantages versus Disadvantages}

Stored value cards will need to be looked at for their benefits to determine if
there are any downfalls in implementing such a system in the university. These
facts and figures need to be looked at closely and then used to reach a final
decision on whether or not this item will be a viable venture to introduce
into the university. In order for this to succeed we need to show that this
possible system is better then the current system in place. The advantages must
far out way the disadvantages of the current system in place. Outlined below
are the advantages and disadvantages of such a system were it be implemented.
Smart cards can perform all the transactions that are now carried out by
magnetic stripe cards plus many other computing tasks. The advantages of such
a card will be looked at here in detail plus examine disadvantages of such a
card.  

\subsection{Benefits}

\items Eliminate the need for multiple cards.

A key benefit is the use of modern smart cards with their ability to be able to
replace common functions of several magnetic stripe cards on a single smart
card.  With this one smart card it can contain one or more credit cards, an
electronic purse, a loyalty program, an electronic signature, act as a benefits
card, be a library card, and any other system that the university wishers to
implement. This would save money, as only one card would need to be made.
\cite{noonan}

\items Greater convenience when purchasing goods.

By adopting the stored value card system, this would allow the University to be
a cash less society as all money transactions would be electronic. This would
reduce the probability of cash theft, as there would be none on the premises.
This would lead to the reduction in the cost of securing, transporting, cash
accounting and insurance cover. \cite{fms}

\items Facilitates the implementation of a rewards program for purchases.

To help make the uptake of cards easier and as an incentive to the students, a
reward system can be implemented similar to what many credit cards have now.
The rewards can be linked to how much a student purchases on campus and
possibly off campus. 

\items Greater security for users utilising campus computers.

Existing computer systems can be modified to provide added login security and
accountability, by only allowing authentication through the use of card readers
on computer terminals and combining this with the use of password or biometric
protection. \cite{irwin}

Once the student's credentials have been verified, it allows for a single
method of authentication which can be used for various on-line services, such
as e-mail sender verification, access to proxy services, etc $\ldots$

\items Allows tracking of the usage of facilities around the university. 	

This would allow the university to see where the highest usage rates are on
such items as computer terminals and printers and with this information be
better equipped to determine where upgrades or rearrangement of services are
needed and would allow better use of resources leading to reduction of cost in
providing such services.

May be used by campus sporting facilities for access to gym/pool. Payment
system may also be integrated at these facilities for users to view their
account details and make payments.

Can be used increased authentication security when accessing secure buildings
and accommodation services as the card can be used as an electronic key for
doors. In adjunction with student accommodation, the card would also provide
access to meals and access to other accommodation facilities. 

Data can be used by university marketing division to create new promotions, and
track success of current promotions - can be used to increase effectiveness of
promotions and increase revenue through value-added offers. \cite{fms}

\subsection{Disadvantages}

\items Potential privacy issues.

There are a number of privacy issues that arise through the implementation of
the stored value card system as aforementioned. Students' use of the card can be
tracked, as well as locations of use, which would allow data mining techniques
to be performed on student activities and profiles on each student to be
created. This leads to the issue of a central location of all ones person
information as the main privacy issue and opposition to such a device being
implemented by various student bodies, such as the Student Representative
Council.

\items Loss of privacy.

For security purposes, a person's smart card might also contain a digital
representation of his or her written signature, or an electronic signature
finger or thumb print information as well as a PIN. The information contained
on the card if stolen would contain a lot of personal information, which
students wouldn't like to be released. \cite{noonan}

Existing computer systems are not equipped to read the stored value cards, thus
they will need to be upgraded to allow for this functionality. The initial cost
outlay may be high; a multi stage roll out over a few years would be needed. The
smart card readers can cost up to \$15 (US) per reader. The cost of a fully
loaded smart card is now \$1.62(US); a magnetic stripe card costs around 50
cents (US). The requirements of the University is approximately 20,000 cards, an
initial capital outlay for smart cards would be \$32,000.00(US), a significant
amount opposed to that of \$10,000.00(US) for simple stripe cards. \cite{irwin}

The card needs to be accepted by the students. It has been shown that for new
technologies to be taken up, they need to be clearly superior to the existing
ways of doing things, and have minimal disadvantages over the current system.
\cite{noonan}

As it would be a one-card system if this card where to be lost you would be 
locked out of every place throughout university. Such as computer labs, gym and
the students room if the card system was was also implemented for the
accommodation units.
