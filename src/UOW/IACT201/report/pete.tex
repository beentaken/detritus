\section{Future expansion and Security considerations}

\subsection{Use of the store value card system for Staff}

Once the adoption of the stored value card system in to the University for
students has been accepted and achieved by both the administration and student
body there exists the opportunity to assimilate the staff information in order
to maintain a single interface to system user information as both students and
staff members would be using similar resources.

\items Moving existing Staff records to the stored value card system.

We can assume that once the student body has begun to use the stored value card
system than a similar argument can be used to move the staff over to a similar
system. It is expected that staff members would have the same needs for privacy
and security as the student body. As the stored value card system has already
been implemented for the student body and the technological infrastructure in
terms of knowledge and plant equipment has already been deployed there would be
a greater return on investment in comparison to the original provisioning.

\items What information would be stored on the card?

As with the student stored value card, the staff stored value card would retain
data which pertains to their duties as a staff member. This information would
have to be encrypted as a matter of course and should only consist of what is
viable to be kept on the card. Information such as authentication credentials
for accessing secured University resources. For data that is not available on
the card, a unique identifier could be kept to allow for access to that
information from an secondary database.

\items What information would not be stored on the card?

There are certain classes of information that should not be kept on the card,
such as wages or salary, performance review or disciplinary actions taken
against the staff member. If these such types of information reside on the card
they could be corrupted through loss, tampering or misadventure. Such an
occurrence could cause havoc as the chain of information would have been
invalidated.

\items Why would expanding the stored value card system to Staff be desirable?

By expanding the use of the stored value card system to staff members, the same
benefits that the student body and administration has been afforded by the
implementation of such a system would spread to employees of the University.
This would reduce the total cost of operations in regards to administering the
information systems pertaining to user of University resources. There is also
the added benefit of having a unified system that is redundant, secure and easy
to use.

\items Objections to the use of stored value cards.

As the use of smart cards to store information are a relatively new technology
and they have been seen in a controversial light with diatribes labelling them
as systems similar to that used by the German Socialist Nazi's in World War II
against the Jewish populations, there would be significant opposition from the
various unions to not implement such a system.

\subsection{Keeping the system security up to date}

\items Potential security risks.

As technology advances the power of computers also increases, such that
cryptographic algorithms that were once considered strong could be compromised
in smaller amounts of time making the use of such antiquated algorithms a
potential security risk. It is anticipated that in the future, especially with
the research and development on quantum computing, that the encryption
mechanisms used on the stored value card system could be rendered trivial,
potentially allowing for the exposure of the private data that is kept on the
cards.

\items Future risk mitigation.

In order for risks, such as the ones outlined above, to have successful plans
for mitigation the University administration needs to pass policies and
delegate authority for the implementation of these policies to ensure that in
the event of the realisation of an identified risk, that the appropriate
security measures can be enacted. In the case of the discovery of an
unidentified risk, that the policy be amended accordingly.
