\input epsf
\input verbatim

\parskip 6 pt

\font\footnotefont=cmr7
\font\mainfont=cmr12
\font\mi=cmti12
\font\mb=cmbx12
\font\headingfont=cmbx16
\font\titlefont=cmbx18

\newcount\itemnum \itemnum=0

\newcount\subitemnum

\def\items{\parindent 24 pt \advance\itemnum by 1 \subitemnum=0 \item{\mb\the\itemnum)}}
\def\subitems{\parindent 24 pt \advance\subitemnum by 1 \itemitem{\mb\the\itemnum.\the\subitemnum)}}
\def\subitemscont{\parindent 24 pt \itemitem{}}

\def\stuff#1{\itemnum=0 {\bf#1:}}

\def\RCS$#1: #2 ${\expandafter\def\csname RCS#1\endcsname{#2}}
\RCS$Id: report.tex 2 2007-07-19 13:00:48Z pdezwart $

\def\title#1#2#3{\centerline{\titlefont #1} \vskip 12pt \centerline{\rm Author: #2} \vskip 12 pt \centerline{\rm Student ID\#: #3} \vskip 12 pt \centerline{\rm \RCSId} \vskip 24pt}

\def\text#1{{\tt #1}}

\def\listing#1#2{\subitems{#1}\par\begingroup\setupverbatim\baselineskip 0pt \input#2 \endgroup}
\def\image#1#2{\subitems{#1}\subitemscont\epsffile{#2}}

\title{ITCS201: Group Report}{Peter Nathaniel Theodore de Zwart}{9840642}

\leftskip=0.75cm
\baselineskip 24pt

\mainfont

\stuff{Glade XSLT}

\items The Glade user interface to building a GUI is well suited to visualising the layout. When it comes to tying all the widgets together, there can be a problem of knowing the heirachy in order to move information from one widget to another. For example:
\subitems In pressing a button and wanting to send a message to a display widget to change what it is viewing.
\subitems Typically this is done through callbacks but the programmer still needs to know where in the heirachy the information is supposed to be sent.

\items One way of being able to view this information would be to transform the Glade XML document in to a visual representation of the heirachy, either in tabular form or in nested form.
\subitems For simplicity, tabular form works best so that deeply nested widgets can be easly looked up.

\items A prototype Glade project was made to show an example of the Glade XSLT described above in action.
\subitems The small project describes a small window with a text entry field and a menu.
\image{The following is a graphical representation of the projects interface:}{window.ps}
\listing{Here is the XML document that describes this interface:}{glade.xml}

\items The XSLT document simply loops through each ``widget'' node and creates a HTML table entry describing:
\subitems The name of the widget.
\subitems The class of the widget, such as: GtkWindow, GtkScrolledWindow, GtkMenuBar, etc $ldots$
\subitems The label of the widget, if any.
\subitems The name of the parent node for the current widget.
\listing{The following is a listing of the XSLT document that achieves the above:}{glade.xsl}
\listing{Here is a listing of the HTML output generated:}{glade.html}
\image{Here is a graphical representation of the HTML output:}{glade.ps}

\bye
