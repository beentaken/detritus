
\font\secfont=cmbx10 scaled \magstep1

\def\section#1 {
	\vskip 12 pt 
	\leftline {
		\secfont #1
	} 
	\nobreak 
}

\parskip 12 pt

\centerline {
	Report on the Evaluation Questionnaires for the Login subsystem.
}

\section {
	Introduction:
}

This document will outline the findings from the \bf User\rm, \bf Observer
\rm\& \bf Expert \rm Evaluation Questionnaires along with some observations
about the walk-through.

\section {
	Walk-through:
}

Unfortunately, the walk-through was missing a crucial new image representing the
dialog box requesting the User to enter their password with the stylus. The
new image had a \it restart \rm button to go back to the beginning of the Login
process.

Apart from this oversight, the walk-through seemed to be brief and clear in its intent.

In future walk-throughs it would be prudent to ensure that they are proof read
before being delivered to the evaluators.

\section {
	User Evaluation Questionnaire:
}

The user evaluation questionnaire was completed in full, with the evaluator
selecting a \it yes \rm for all questions with a ``yes'' or ``no'' answer. This
was contrary to what was originally planned with each question having a scale
between 1 to 5, with 1 representing ``very poor'' and 5 representing ``very
good''. Because of this, there is not as much information to be gleaned from
this questionnaire as could have been garnered from one with a finer granularity
in how the questions could have been answered.

Bear in mind that this questionnaire is intended to retrieve feedback from a
\it novice \rm user, therefore the questions posed and their restricted
set of answers did reduce the complexity of the questionnaire.

There was one question that used the 5 point scale. However, it was a catch all
question, rating the system as a whole, in which it was rated as \it good \rm.

The finding of this report in regards to the user evaluation is to ensure that
all future user evaluation questionnaires allow a greater freedom of answers,
allowing the user evaluator to give us more feedback. Perhaps a simple area
for written feedback would have been appropriate given the scale of the Login
sub-system. As the user evaluation of the Login sub-system was positive all
round, there was no visible improvements that could be drawn from this
questionnaire.

\section {
	Observer Evaluation Questionnaire:
}

The observer evaluation questionnaire, completed in full, gave some more
information compared to the user evaluation questionnaire, in respect to how
the user evaluator interacted with the walk-through of the Login sub-system.

Being a questionnaire for the observer of the user evaluator, the questions
posed were directed more towards the apparent responses of the user to various
user interface stimuli. We did gain a useful insight in how not to deliver
a walk-through, as the observer evaluator reported that the aforementioned
``restart'' button was not apparent.

Once again, we received an outstanding evaluation. Leading to the thought that
perhaps the questions were either too simple or misleading. It would have been
good to have had a small free form comment field to get feedback on areas not
thought of or covered by this questionnaire.

There is an improvement in this questionnaire in comparison to the previous one.
The observer evaluator was able to respond on the areas the user may have
been having trouble with by writing the section(s) in a small text field, as the
user encountered them.

From the above, there were no direct improvements that could be constructed from
the observer evaluation questionnaire.

\section {
	Expert Evaluation Questionnaire:
}

In the expert evaluation questionnaire, the evaluator gave the interface a good
rap, however, the available answers to the questions posed were limited in
scope similar to that of the user evaluation questionnaire.

Because of the oversimplification of the expert evaluation questionnaire, very
little constructive information was able to be found and the evaluators
themselves may have been underutilised and perhaps offended at the
questionnaires apparent lack of probing questions.

For similar reasons outlined in the report of the user evaluation questionnaire
in regards to the large percentage of questions having only ``yes'' or ``no''
answers, this questionnaire could only deliver limited information.

There was one article of fresh insight in to how the design of the Login
sub-system could be improved, in where the user is having their thumb print
scanned, the user is not informed on how long the process will take and the user
has no course for ceasing the process.

In respect to this one boon from the expert evaluator, it would be ideal if
the phase in which the users thumb print is being scanned have the following
elements:

\item{$\bullet$}	A 60 second timeout if no input is received by the
			biometric device.
\item{$\bullet$}	A error dialog indicating that an incomplete biometric
			scan was obtained, requiring the user to repeat this
			process.

\section {
	Conclusion:
}

It is the findings of this report that in the future, the \bf user \rm
evaluation questionnaire is to be kept in a similar simple form with the
inclusion of a small text field for comments. That the \bf observer \rm
evaluation questionnaire contain more areas for written feedback as this was
the medium in which we received the most useful information. Lastly, the
\bf expert \rm evaluation questionnaire must contain pertinent questions about
the interface of a system, commensurate with the level of prowess of an expert
evaluator. In light of the failings of the questionnaires, the Login sub-system
itself appeared to need very few changes, with those changes required being a
logical extension of the original design.

\bye
