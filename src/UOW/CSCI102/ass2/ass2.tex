\def\RCS$#1: #2 ${\expandafter\def\csname RCS#1\endcsname{#2}}
\RCS$Revision: 1.8 $
\RCS$Date: 2003/05/07 00:12:03 $

\parskip = 12 pt

It can be shown that the Rational Unified Process model of software engineering is superior in comparison to the Waterfall model in the way it mitigates risks early and adequately understands the reality of changing user requirements.

The Waterfall model consists of three phases of development: Analysis, Design and Implementation. The Analysis phase consists of determining user requirements, Design phase is the crafting of a solution to meet the user requirements and the Implementation phase houses the coding and testing of the work units that facilitate the design. The problem here is that there is no method for reviewing the project as it moves along. This does not allow the user to gain a cognitive perspective of their needs in relation to the project. The Waterfall model assumes that all user requirements are fixed from the inception of the project.

In the Rational Unified Process however, the project moves along through various iterative phases. The Inception, Elaboration, Construction and Transition phases.

In the Inception phase, a ``vision of the end product'' (Atkinson et al., 2002, p 18) is determined, meaning that there is no concrete definition of requirements, as opposed to the Waterfall model, leading to the idea that the product can be viewed from the users perspective as amorphous, to a certain extent, changing as light is gradually shed on the software over the entirety of the projects tenure.

The Elaboration phase describes the functional requirements of the project and a suitable plan is created, in comparison to the Waterfall model where the project plan and functional requirements are based upon a fixed view of the user requirements, the Rational Unified Process draws strength on the realisation that the user does not fully understand their needs at the projects Inception. This phase takes care of this by utilising the preconceived vision and rather than setting a deterministic set of functions, a minimum set of functional requirements is created from currently known user requirements. This set will change over time as user comes to greater cognition of their functional needs.

Afterwards, in the Construction phase, the majority of the code base is conceived in iterative cycles, where each cycle is a miniature project within itself. However this miniature project has a limited scope, of adding a small portion of the functionality to satisfy the functional requirements and is similar to the workings of a Waterfall modelled project. In the sense that each iteration has an analysis, design and implementation phase. At the end of each iteration, a review of the work is done to ensure that user understands and agrees that the project is heading towards the vision created in the Inception phase. If it is deemed that the iteration has not done so, the team must reiterate this step to close the gap upon the project and the vision. Each successive iteration aims to add more functionality upon the previous iteration.

Finally, in the Transition phase, the project undergoes it's beta tests and if successful, the users are trained in the care and feeding of the product. Assuming that all goes well, the final product is released to the user. Otherwise, another iteration may have to to be undertaken to hone the product.

As it has been shown the Rational Unified Process can adapt to the reality of changing user requirements as opposed to the Waterfall model that takes a naive approach. The realistic approach of the Rational Unified model allows the project team to deal with the risk of developing a product that does not fit the users requirements when the product is released. This ensures that both the user and project team is satisfied with the end result.

\eject 

\item{1.} Atkinson, C. et al. 2002, \it Component-based Product Line Engineering with UML\rm, Addison-Wesley, Great Britain.
\item{2.} Wilkie, G. 1994, \it Object-Oriented Software Engineering\rm, Addison-Wesley Publishing Company, Cambridge.
\item{3.} Tate, G. et al. 1995, \it Software engineering: practice, management, improvement\rm, Addison-Wesley, Australia.
\item{4.} Walker, R. 1998, \it Software project management: a unified framework\rm, Addison-Wesley.

\bye
