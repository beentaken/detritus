\input epsf
\input verbatim

\parskip 6 pt

\font\titlefont=cmbx18

\newcount\itemnum \itemnum=0

\newcount\subitemnum

\def\items{\parindent 24 pt \advance\itemnum by 1 \subitemnum=0 \item{\bf\the\itemnum)}}
\def\subitems{\parindent 24 pt \advance\subitemnum by 1 \itemitem{\bf\the\itemnum.\the\subitemnum)}}
\def\subitemscont{\parindent 24 pt \itemitem{}}

\def\stuff#1{\itemnum=0 {\bf#1:}}

\def\RCS$#1: #2 ${\expandafter\def\csname RCS#1\endcsname{#2}}

\def\title#1#2#3{\centerline{\titlefont #1} \vskip 12pt \centerline{\rm Author: #2} \vskip 12 pt \centerline{\rm Student ID\#: #3} \vskip 12 pt \centerline{\rm \RCSId} \vskip 24pt}

\def\text#1{{\tt #1}}

\def\listing#1{\subitems{Listing of #1:}\par\begingroup\setupverbatim\input#1 \endgroup}
\def\image#1{\subitemscont\epsffile{#1}}

\title{CSCI399: Assignment 2}{Peter Nathaniel Theodore de Zwart}{9840642}

\stuff{Introduction}

\items	For assignment 2 in CSCI399: Server Technology, we have been challenged to implement a web based version of the {\it classic} game ``Othello''. Given the resources available to us at University and what I have at home, I decided to go with what I had at home, as I already had \text{Apache} with \text{PHP4} and \text{PostgreSQL} set up on a Quad PentiumPro.

\items	The game had a few design goals:

\subitems	Use \text{PHP} for the server side scripting.
\subitems	Use a relational database for the storage of player information.
\subitems	Using the above database, have mechanisms in place for player authentication.
\subitems	Have the players pit themselves against a simple computer player.

\items	The first consideration was to design the database table that would be used.

\items	For the readers perusal, a user has been created for them to play in the Othello league.

\subitems	URI: \text{http://froob.optus.nu:65533/\~{}dezwart/ass2/othello.php}
\subitems	Username: \text{REDACTED}
\subitems	Password: \text{REDACTED}

\stuff{SQL considerations}

\items	As I was using \text{PostgreSQL} rather than \text{Oracle} as most other people were using, I aimed to keep the table design as close to the SQL specification as possible.

\subitems For instance, not using the \text{text} field type for columns such as \text{name} or \text{passwd}, as this is non-standard although handy.

\listing{table.sql}

\items	A constraint was placed upon the \text{won} column that it couldn't be greater than the \text{played} column, meaning that no player could win more than they played.

\items	There were some security issues that had to be sorted out with access to the SQL database as the web server was running as user \text{www-data} so their database user had to have rights granted to them to allow INSERT, SELECT and UPDATE privileges.

\stuff{Common Code}

\items	Partway through implemented the skeletons of the various web pages that would comprise the Othello game, it became apparent that a lot of the authentication and HTML markup could be percolated in to their own files that would be required in to the resulting PHP documents.

\items	The common HTML markup preamble and postamble was put in \text{common.php} which all pages would require to load.

\listing{common.php}

\items	From \text{common.php} another document was included, which comprised the page authentication mechanisms which received a user name and a password from the client's web browser and then checked that information with the relevant columns in the database.

\listing{auth.php}

\items	Here is a window capture of the authentication dialog from \text{Mozilla 1.6}:

\image{auth.ps}

\stuff{Main page}

\items	The main page had to have two elements.

\subitems	A link to the page that showed the standings of individual players in the league.
\subitems	A link to the page that had the game.

\listing{othello.php}

\subitems	Following is a window capture of the Main page, \text{othello.php}:

\image{othello.ps}

\stuff{League listing page}

\items	The league listing page had to query the database and produce a table containing data about all the players. This table had to have:

\subitems	The name of the player.
\subitems	The number of games played by the player.
\subitems	The number of games won by the player.

\items	As it can be seen, I've created a \text{computer} player, to record the relative position of the computer player versus the human players. This saves people having to do the addition themselves.

\items	The implementation of this page was straightforward, using the HTML markup preamble function and then defining a table and using a loop to print out table row equivalents from an SQL query.

\listing{league.php}

\subitems	Following is a window capture of the league listing page before playing a game, as you can see, the computer has already managed to win a game against one of the players.

\image{league-pregame.ps}

\subitems	Later on in this report, another window capture of this page will be included depicting the results after a game.

\stuff{Game page}

\items	The game page has some interesting hidden elements, such as that it uses cookies to retain board state between pages. Apart from that, it's just a table of images with form information tied to each table cell.

\items	When the game is started, usually by a simple \text{GET} request to the game page \text{game.php}, the state information is to be set up and the initial board displayed.

\items	The logic behind game creation and continuation checks to see if there is already a cookie available for game play.

\subitems	If no cookie, then start the game.

\subitems	Otherwise, continue using the old cookie..

\subitems	Here is a window shot of the initial board setup:

\image{game-started.ps}

\items	Then a player would simply click on where they wanted to place their next token and a \text{POST} request would be issued to \text{game.php} with the index number of the selected table cell.

\subitems	After a while, the page would look akin to this one:

\image{game-middle.ps}

\subitems	However, if an illegal move was attempted, then you may notice that the table caption changes:

\image{game-illegal-move.ps}

\items	Eventually, depending on your skill playing Othello, mine, if expressed as a probability of winning, is still stuck around the zero probability horizon.

\subitems	When you do finish a game, the screen will change slightly to reflect the state at the end, indicating if you won or lost and where you can go from there.

\subitems	Here is a window capture of where I lost. As per usual:

\image{game-lost.ps}

\items	From the two exit points from a game terminus, the league listing page can be viewed again to see what the previous game has done to your record.

\subitems	As can be seen, I lost, again, surprise, and the computer won. Here is another window shot depicting this:

\image{league-postgame.ps}

\subitems	Thus, both \text{played} columns have been incremented by one for both myself and the computer player.

\subitems	Also, the pesky computer has had it's \text{won} column incremented. ``I'll get you next time gadget, next time...''

\items	From an implementation point of view, the game page was split in to two PHP files, one for the HTML portion and one for the game logic part.

\items	The HTML script does the checking to see if we need to set up a new game or if there is currently a game in session.

\listing{game.php}

\items	The game logic script is an class that has the code to ensure that the player has entered in a valid move and then generates the computers response.

\items	It was based on the JAVA implementation of a text based Othello game by Roxanne Canosa, Dr. Sandeep Mitra, and Dr. Leslie Lander of the SUNY-Brockport Java Interest Group.

\items	I found their implementation somewhat, unclean, in terms of design and implementation, so using their algorithm, created a PHP version which doesn't resemble spaghetti so much.

\items	The algorithms was also altered to only check one player move and generate the computers response, unless the player isn't able to move, where the computer will move until a situation presents itself that the player can, more than likely, the player will just be bludgeoned in to submission.

\listing{computer.php}

\bye
