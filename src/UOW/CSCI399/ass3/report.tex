\input epsf
\input verbatim

\parskip 6 pt

\font\titlefont=cmbx18

\newcount\itemnum \itemnum=0

\newcount\subitemnum

\def\items{\parindent 24 pt \advance\itemnum by 1 \subitemnum=0 \item{\bf\the\itemnum)}}
\def\subitems{\parindent 24 pt \advance\subitemnum by 1 \itemitem{\bf\the\itemnum.\the\subitemnum)}}
\def\subitemscont{\parindent 24 pt \itemitem{}}

\def\stuff#1{\eject \itemnum=0 {\bf#1:}}

\def\RCS$#1: #2 ${\expandafter\def\csname RCS#1\endcsname{#2}}
\RCS$Id: report.tex 2 2007-07-19 13:00:48Z pdezwart $

\def\title#1#2#3{\centerline{\titlefont #1} \vskip 12pt \centerline{\rm Author: #2} \vskip 12 pt \centerline{\rm Student ID\#: #3} \vskip 12 pt \centerline{\rm \RCSId} \vskip 24pt}

\def\text#1{{\tt #1}}

\def\listing#1{\subitems{Listing of #1:}\par\begingroup\setupverbatim\input#1 \endgroup}
\def\image#1{\subitemscont\epsffile{#1}}

\title{CSCI399: Assignment 3}{Peter Nathaniel Theodore de Zwart}{9840642}

\stuff{Introduction}

\items For CSCI399: Assignment 3, we were given the task to create a Nutrition Clinic patient consultation management system, using Java Servlets, where:
\subitems Patients could be added to the system.
\subitems Consultation data could be entered for a patient.
\subitems Patient data could be displayed in tabular or graphical format.
\subitems Patients could be listed.
\subitems Patients could be removed from the system.

\items Certain roles also had to be defined, which allowed for authenticated access to patient data.
\subitems Administrators \text{(administrator)}: were allowed to delete patients.
\subitems Nutritionists \text{(nutritionist)}: were allowed to add and view patients and their data.
\subitems Clerks \text{(clerk)}: were allowed to enter in patient consultation data.

\items To achieve this, the system needed to have four distinct servlets:
\subitems New Patient: for adding a new patient.
\subitems Consultation: for adding consultation data for a patient.
\subitems View Patient Data: for view patient data.
\subitems View Patients: for view patients and optionally removing them.

\stuff{System configuration}

\items In order for the system to behave as desired, there was some footwork that had to be done before the servelets could be created.

\items The SQL tables needed to be set up to describe the patients and their consultation data.
\subitems There were two tables, one for a patient and one for patient consultation data.
\subitems The patient table needed to have, per patient, a system allocated identifier, a surname, patient's initials, age, gender and height.
\subitems The consultation data table needed to have, per consultation, the system identifier of the patient, the date of the consultation and the patient's weight.
\listing{tables.sql}
\subitems The tables were set in a way that when a patient was removed form the patient table, their corresponding consultation entries would be removed via a cascading delete.

\items Once the SQL setup was done, the HTML infrastructure needed to be set up. To allow for future expansion and for consistency of the component web pages, a cascading style sheet was set up.
\subitems The CSS needed to have provisioning for error messages and other informative sections.
\listing{style.css}

\items Tomcat then needed to have some users set up to take advantage of Jakarta's authentication model.
\subitems ``canker'' is the clerk.
\subitems ``aardvark'' is the administrator.
\subitems ``nuncle'' is the nutritionist.
\listing{tomcat-users.xml}

\items To set up the servlets so that Tomcat knew what to serve up and how, a \text{web.xml} was created to allow for the following:
\subitems Definition of all the four servlets.
\subitems Access controls for the servlets.
\subitems Authentication mechanisms.
\subitems Database information.
\listing{WEB-INF/web.xml}

\items Some metapages were created to facilitate user authentication, login errors and to tie all the servlets together:
\subitems A main index page was created to tie all the various servlets together.
\listing{index.html}
\subitems A simple login page used to authenticate the user.
\listing{login.html}
\subitems An even simpler error page if the user didn't get authenticated.
\listing{error.html}

\stuff{Servlet Common Code}
\items Within lies all the common code used by the various servlets. Notably:
\subitems SQL Authentication.
\subitems SQL Query.
\subitems HTML Preamble and Postamble.
\listing{WEB-INF/classes/CommonServlet.java}

\stuff{New Patient Servlet}
\items Commented with Javadoc.
\listing{WEB-INF/classes/NewPatientServlet.java}

\stuff{New Consultation Servlet}
\items Commented with Javadoc.
\listing{WEB-INF/classes/NewConsultationServlet.java}

\stuff{Review Servlet}
\items Commented with Javadoc.
\listing{WEB-INF/classes/ReviewServlet.java}

\stuff{Administration Servlet}
\items Commented with Javadoc.
\listing{WEB-INF/classes/AdminServlet.java}

\bye
