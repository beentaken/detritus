\font\sixrm=cmr6
\font\mainfont=cmr10
\font\mi=cmti10
\font\mb=cmbx10
\font\sectionfont=cmbx12
\font\headingfont=cmbx12
\font\titlefont=cmbx12

\def\heading#1{\vskip 12 pt \leftline{\headingfont #1}}

\newcount\footnotes \footnotes=0

\def\footnoter#1{\advance\footnotes by 1 \footnote{$^{\the\footnotes}$}{\mainfont #1}}

\def\section#1{\vskip 12 pt \leftline{\sectionfont #1}}

\def\title#1{\centerline{\titlefont{#1}} \footline{\hss\tenrm{#1:} Page \folio\hss}}

\def\pskip{\parskip 12 pt}
\def\lskip{\parskip 3pt}
\def\sskip{\parskip 6pt}

\pskip
\parindent 24 pt

\title{Interconnecting Cisco Networking Devices - Part 1}

\mainfont

\heading{Module 1 - Building a simple network}

Module Objectives

\item - Identify the benefits of computer networks \& how they function.
\lskip
\item - Identify common threats to a network \& thread-mitigation methods.
\item - Identify \& compare the communication models that control host-to-host communications.
\item - Describe IP\footnoter{IP - Internet Protocol} address classification \& how a host can obtain an address.
\item - Describe the process that TCP\footnoter{TCP - Transmission Control Protocol} uses to establish a reliable connection.
\item - Describe the host-to-host packet delivery process.
\item - Describe how Ethernet operates at Layer 1 and Layer 2 of the OSI\footnoter{OSI - Open Systems Interconnection} model.
\item - Identify metohds of connecting to an Ethernet LAN\footnoter{LAN - Local Area Network}.
\pskip

Lesson 1 - Exploring the Functions of Networking

Objectives

\item - Define a network.
\lskip
\item - List the common components of a network.
\item - Interpret network diagrams.
\item - List major resource-sharing functions of networks \& their benefits.
\item - List four common user applications that require network access \& the benefits of each.
\item - Describe the impact of user applications in the network.
\item - List the categories of characteristics that are used to describe the various network types.
\item - Compare \& contrast logical \& physical topologies.
\item - List the characteristics of a bus topology.
\item - List the characteristics of a star \& extended-star topology.
\item - List the characteristics of a ring \& dual-ring topology.
\item - List the characteristics of a mesh \& partial-mesh topology.
\item - Describe the methods of connecting to the Internet.
\pskip

What is a Network?

A network is a connected collection of devices \& end systems, such as computers \& servers, which can communicate with each other.

\item - Main office, where most corporate information is located.
\lskip
\item - Branch office, local network resources such as printers, most information accessed from main office.
\item - Home office, require on-demand connection to main office or branch office to access information or use network resources.
\item - Mobile users, connect to main office when at main office, branch office when at branch office, \&c... Location of mobile user determines access requirements.
\pskip

Common Physical Components of a Network.

Five major catagories:

\item - PCs, endpoints in the network, send \& receive data.
\lskip
\item - Interconnections, components that provide means for data to travel from one point to another point in the network.
\itemitem - NICs\footnoter{NIC - Network Interface Card}, translate computer data to format transmissable over local network.
\itemitem - Network media, cables or wireless media that provides means for signals to be transmitted from one network device to another.
\itemitem - Connectors, provide connection points for network media.
\item - Switches, provide network attachment to end systems \& intelligent switching of data within the local network.
\item - Routers, interconnect networks \& choose best path between networks.
\item - WLAN\footnoter{WLAN - Wireless LAN}, connect endpoints without using cables.

\bye
